\documentclass[a4paper, 11pt,titlepage, openright, twoside]{report}
\usepackage[utf8]{inputenc}
\usepackage[T1]{fontenc}
\usepackage{silence}
\usepackage{amsmath,amsfonts,amssymb,amsthm}
\usepackage{bm}
\usepackage{mathtools}
\usepackage[inline]{enumitem}
\usepackage{newunicodechar}
\usepackage[margin=3cm,bindingoffset=1cm]{geometry}
\usepackage{stmaryrd}
\SetSymbolFont{stmry}{bold}{U}{stmry}{m}{n}
% https://tex.stackexchange.com/a/106719
\DeclareSymbolFont{sfletters}{OML}{cmbrm}{m}{it}
\usepackage[nopatch=footnote]{microtype}
\usepackage[dvipsnames]{xcolor}
\usepackage{mathpartir}
\usepackage{biblatex}
%\usepackage{cleveref}
\usepackage{tikz}
\usepackage{tikz-cd}
\usepackage{listings}
\lstset{basicstyle=\ttfamily}

\usepackage[]{hyperref}
\usepackage[all]{hypcap}
\setlength{\parskip}{0pt}

\addbibresource{refs.bib}


\title{\textbf{A Fine Calculus for Static Delimited Control}}
\author{Wiktor Kuchta}

\date{60 września 2024} %TODO

\usepackage{titling}

\renewcommand \maketitlehookb {
  \begin{center}\large
  Fajny rachunek dla statycznie ograniczonych operatorów sterowania
  \end{center}
  \vfil
}

\renewcommand \maketitlehookc {
  \vfil
  \begin{center}
  \large Praca magisterska \\[0.85em]
  \begin{tabular}[t]{rl}
  \textbf{Promotor:} & dr hab. Dariusz Biernacki
  \end{tabular}\end{center}
  \vfil\vfil\vfil\vfil
  \begin{center}Uniwersytet Wroc\l{}awski\\
  Wydzia\l{} Matematyki i Informatyki\\
  Instytut Informatyki
  \end{center}
}


\newcommand{\true}{\textsf{true}}
\newcommand{\false}{\textsf{false}}
\newcommand{\shiftz}{\textsf{shift0}}
\newcommand{\abort}{\textsf{abort}}
\newcommand{\keyword}[1]{\textsf{\textup{#1}}}
\newcommand{\KwOp}{\keyword{op}}
\newcommand{\Op}{\KwOp\,}
\newcommand{\KwHandle}{\keyword{handle}}
\newcommand{\Handle}{\KwHandle\;}
\newcommand{\KwWith}{\keyword{with}}
\newcommand{\With}{\;\KwWith\;}
\newcommand{\KwRaise}{\keyword{raise}}
\newcommand{\Raise}{\KwRaise\;}
\newcommand{\Ask}{\textsf{ask}}
\newcommand{\KwTry}{\keyword{try}}
\newcommand{\Try}{\KwTry\;}
\newcommand{\KwLet}{\keyword{let}}
\newcommand{\Let}[3]{\keyword{let}\;#1\;\keyword{=}\;#2\;\keyword{in}\;#3}
\newcommand{\RLet}[3]{\Let{#1}{\raisebox{0.5 ex}{$#2$}}{#3}}
\newcommand{\KwLift}{\keyword{lift}}
\newcommand{\Lift}[1]{\KwLift\;#1}
\newcommand{\subst}[2]{\{#1{:=}#2\}}
\newcommand{\E}{\mathcal{E}}
\newcommand{\K}{\mathcal{K}}
\renewcommand{\S}{\mathcal{S}}
\newcommand{\A}{\mathcal{A}}
\newcommand{\kT}{\mathsf{T}}
\newcommand{\kE}{\mathsf{E}}
\newcommand{\kR}{\mathsf{R}}
\newcommand{\Free}{\textrm{-}\mathrm{free}}
\newcommand{\Obs}{\mathrm{Obs}}
\newcommand{\N}{\mathbb{N}}
\DeclareMathOperator{\dom}{dom}
\newcommand{\+}{\enspace}
\newcommand{\lStr}{\textsf{Str}}
\newcommand{\lPar}{\textsf{Par}}

\newtheorem{corollary}{Corollary}
\newtheorem{lemma}{Lemma}
\newtheorem{theorem}{Theorem}
\newtheorem{prop}{Proposition}


\newunicodechar{│}{\mid} % Digr vv
\newunicodechar{╱}{\mathbin{/}} % Digr FD
\newunicodechar{∷}{::} % Digr ::
\newunicodechar{□}{\square} % Digr OS
\newunicodechar{∅}{\emptyset} % Digr /0
\newunicodechar{α}{\alpha}
\newunicodechar{β}{\beta}
\newunicodechar{δ}{\delta} % Digr d*
\newunicodechar{ε}{\varepsilon}
\newunicodechar{γ}{\gamma} % Digr g*
\newunicodechar{ι}{\iota} % Digr i*
\newunicodechar{κ}{\kappa}
\newunicodechar{λ}{\lambda}
\newunicodechar{μ}{\mu}
\newunicodechar{ν}{\nu}
\newunicodechar{ρ}{\rho}
\newunicodechar{σ}{\sigma}
\newunicodechar{τ}{\tau}
\newunicodechar{η}{\eta} % Digr y*
\newunicodechar{Δ}{\Delta}
\newunicodechar{Γ}{\Gamma}
\newunicodechar{Ω}{\Omega} % digr W*
\newunicodechar{ℕ}{\N} % Digr NN 8469 nonstandard
\newunicodechar{⊆}{\subseteq} % Digr (_
\newunicodechar{∪}{\cup} % Digr )U
\newunicodechar{∈}{\in} % Digr (-
\newunicodechar{∃}{\exists} % Digr TE
\newunicodechar{∀}{\forall} % Digr FA
\newunicodechar{∧}{\wedge} % Digr AN
\newunicodechar{∨}{\vee} % Digr OR
\newunicodechar{⊥}{\bot} % Digr -T
\newunicodechar{⊢}{\vdash} % Digr \- 8866 nonstandard
\newunicodechar{⊨}{\models} % Digr \= 8872 nonstandard
\newunicodechar{⊤}{\top} % Digr TO 8868 nonstandard
\newunicodechar{⇒}{\implies} % Digr =>
\newunicodechar{⇔}{\iff} % Digr ==
\newunicodechar{↦}{\mapsto} % Digr T> 8614 nonstandard
\newunicodechar{≠}{\neq}
\newunicodechar{⟦}{\bm{[}}
\newunicodechar{⟧}{\bm{]}}
\newunicodechar{≥}{\ge}
\newunicodechar{≤}{\le}
\newunicodechar{≡}{\equiv}
\newunicodechar{≈}{\approx}


% cursed
\WarningFilter{newunicodechar}{Redefining Unicode}
\newunicodechar{·}{\ifmmode\cdot\else\textperiodcentered\fi} % Digr .M
\newunicodechar{×}{\ifmmode\times\else\texttimes\fi} % Digr *X
\newunicodechar{→}{\ifmmode\rightarrow\else\textrightarrow\fi} % Digr ->
\newunicodechar{←}{\ifmmode\leftarrow\else\textleftarrow\fi} % Digr <-
\newunicodechar{…}{\ifmmode\dots\else\textellipsis\fi} % Digr .,
\newunicodechar{⟨}{\bm{\langle}}
\newunicodechar{⟩}{\bm{\rangle}}

\begin{document}

\maketitle


\thispagestyle{empty}
\cleardoublepage
\begin{abstract}
	We consider a variant of the call-by-value $λμ$-calculus extended with control delimiters,
	in which $μ$ becomes the static delimited control operator shift0.
	We propose new reduction rules for cases where the captured continuation can be determined statically.
	We prove the calculus confluent and adequate wrt.\,operational semantics.
	We then let delimiters carry data (resulting in a form of dynamic binding), which lets us express deep effect handlers.
	The new reductions cooperate with the encoding to let us statically handle effects.
	We also propose natural encodings of return clauses for delimiters.
%	We propose purity assertions for lightweight purity-based reasoning.
%	We prove the reduction is confluent and adequate wrt.\ operational semantics.
	We argue that the calculus is stronger than previous results in the literature and could form basis for
	reasoning and optimizations for the aforementioned forms of control.
	\begin{center} \rule[3pt]{300pt}{1pt} \end{center}
	polski
\end{abstract}


\thispagestyle{empty}
\cleardoublepage
\setcounter{page}{5}
\tableofcontents


\chapter{Introduction}
Consider a program of the following form in a language with (unnamed) exceptions:
\begin{align*}
	&\Try \\
	&\quad \Let{y}{e}{\Raise (y*2)} \\
	&\KwWith\; x.\,x*3
\end{align*}
If the execution gets to $\Raise (y*2)$, the handler will catch the exception and produce the result $y*2*3$.
%$$\Try \Let{x}{e}{\Raise (2*x)} \With z.\,3 * z$$
%We know that \textit{the} $\KwRaise$ (if we get to it) is going to be handled by \textit{the} handler.
%And in that case, $x*2$ is going to be multiplied by $3$.
How to \textit{optimize} the program, so that we only perform one multiplication, $y*6$?

We can ask the same question for \textit{effect handlers}.
This generalization allows us to capture the \textit{continuation} of the \textit{operation} (a.k.a.\,\textit{generic effect}) we are handling
and bind it as a variable.
The conventional evaluation step rule is:
$$\Handle E[\Op v] \With x\;k.\,e ↦  e\subst{x}{v}\subst{k}{λz.\,\Handle E[z] \With x\;k.\,e} $$
This is a form of \textit{static delimited control}: control because we capture continuations (represented with the help of
\textit{evaluation contexts} $E$),
delimited because the continuation is delimited by the handler, and static because the delimiter remains
in the captured continuation.
%The last part is crucial.

Here's a similar example:
$$ \Handle \Let{y}{e}{\Op (y*2)} \With x\;k.\,k\,(x*3) $$
Just as in the example of exceptions, we know that $\Op (y*2)$ will be handled
by the $x\;k.\,k\,(x*3)$ handler -- even if continuation capture occurs during the execution of $e$,
the $\KwOp$ and the $\KwHandle$ will stay together.

We therefore know the substitution that will be performed during the handling of the operation:
$x:=y*2$.
Exactly this substitution is needed to rewrite the two multiplications into one.
We also can deduce the substitution $k:=λz.\,\Handle z \With x\;k.\,k\,(x*3)$.
But where to substitute?
Certainly not in the handler's clause,
because the clause could be used for different operations inside $e$,
and $y$ is not even in scope.

Since we know the clause that will run for the specific operation,
instead of performing the operation
we could \textit{abort} the handler and execute the clause specialized with the specific substitutions.
That is, rewrite the $\Op (y*2)$ into:
$$\abort\,(k\,(x*3))\subst{x}{y*2}\subst{k}{λz.\,\Handle z \With x\;k.\,k\,(x*3)}$$
%The result of the substitution and the following simplification are:
%$$\Handle \Let{x}{e}{\abort\,\big((λy.\,\Handle y \With z\;k.\,k\,(z*3)) (x*2*3)\big)} \With z\;k.\,k\,(z*3)$$
After some simplification we obtain:
$$\Handle \Let{y}{e}{\abort\,(y*6)} \With x\;k.\,k\,(x*3)$$

\noindent{}
\hrulefill

\noindent{}
The preceding whirlwind development may raise some questions.
\begin{enumerate}
	\item Is it even correct?
	\item Where does the $\abort$ come from?
	\item Will the optimization rule fit on the page, if the evaluation one almost didn't?
\end{enumerate}
We will try to address them.

%Of course, this poses questions such as ``where does the abort come from?''
%and ``how monstrous will the general rewrite rule be?''...

The issue with conventional syntax for operations and effect handlers is that
the operations are \textit{passive}.
In a real implementation, when we get to an operation, we immediately begin the search for a handler.
The syntax does not reflect that – it doesn't say what happens next. It's just an operation name and a payload value.

We will therefore base our calculus for optimizations on the other static delimited control operator:
$\shiftz$. The conventional rule for $\shiftz$ and its delimiter \textsf{reset} (written $⟨·⟩$) is:
$$⟨E[\shiftz\,k.\,e]⟩ ↦ e\subst{k}{λy.\,⟨E[y]⟩}$$
Here, the $\shiftz$ assumes the active role by initiating capture of continuation $k$ and carrying the expression $e$ that says what to do with it,
while the delimiter is passive -- it merely delimits.
We now express \abort{} simply as a $\shiftz$ that ignores the continuation: $\textsf{abort}\,e ≡ \shiftz\,\_.\,e$.

To express effect handlers, we will extend delimiters by letting them carry a value.
The value will be a lambda abstraction $λx\,k.\,e$ representing the handler's clause.
An operation will be a $\shiftz$ that explicitly asks for it and applies the payload and continuation.
So, the $\abort$ arises as a result of substitution for $k$ in the $\shiftz$'s body.

Finally, to achieve fine-grained steps instead of complex rules that shuffle entire contexts around,
we will get rid of the irregularity that is the $λ$-abstraction wrapping the continuation.
Delimited continuations will not be represented as call-by-value functions, but instead
as their own sort with the ability to plug in any expression, not just a value.
This lets us use structural substitution known from the $λμ$-calculus and all the benefits it brings \cite{benefit}.

\chapter{Syntax}
\label{Syntax}

\begin{figure}
\begin{align*}
	\text{(value) variables} &&x,y,... \\
	\text{continuation variables} && κ \\
	\text{values} &&v,u &::= x │ λx.\,e \\
	\text{expressions} &&e,t &::= v │ v\;v │ \Let{x}{e}{t} │ \S κ.\,e │ ⟨e / v⟩ │ κ⟦e⟧ │ \Ask(κ) \\
	\text{evaluation contexts}&&E   &::= □ │ \Let{x}{E}{t} \\
	\text{tail contexts}&&L &::= □ │ \Let{x}{e}{L} \\
	\text{metacontinuations}&&K   &::= □ │ \Let{x}{K}{t} │ κ⟦K⟧ │ ⟨K/v⟩
\end{align*}
\caption{Syntax.}
\label{syntax}
\end{figure}
\begin{figure}
\begin{align*}
	(λx.\,e)\;v &↦ e\subst{x}{v} \\
	\Let{x}{v}{t} &↦ t\subst{x}{v} \\
	⟨v/u⟩ &↦ v \\
	⟨E[\S κ.\,e]/v⟩ &↦ e\subst{κ}{⟨E/v⟩} \\
	κ⟦E[\S κ'.\,e]⟧ &↦ e\subst{κ'}{κ⟦E⟧}
\end{align*}
\caption{Evaluation step.}
	\label{step}
\end{figure}
\begin{figure}
	\begin{align*}
		λx\,y.\,e &≡ λx.\,λy.\,e & \text{multi-param lambda} \\
%		e\;v &≡ \Let{x}{e}{x\;v} & \text{application to expr}\\
		e_1\;e_2 &≡ \Let{x}{e_1}{\Let{y}{e_2}{x\,y}} & \text{left-to-right application}\\
		\A\,e ≡ \abort\,e&≡ \S \_.\,e & \text{abort}\\
		⟨e⟩ &≡ ⟨e/λx.\,x⟩ & \text{reset with dummy value}\\
		\shiftz\,k.\,e &≡ \S κ.\,\Let{k}{λx.\,κ⟦x⟧}{e} & \text{value-binding shift0}\\
		⟨e│y.\,t_r⟩ &≡ ⟨\Let{y}{e}{\A\,t_r}⟩ & \text{reset with return clause} \\
		\Op v &≡ \S κ.\,\Ask(κ)\;v\;(λx.\,κ⟦x⟧) & \text{operation/generic effect}\\
		\Handle e \With x\,k.\,t &≡ ⟨e/λx\,k.\,t⟩ & \text{effect handler}\\
		\Handle e \With x\,k.\,t │ y.\,t_r &≡ ⟨\Let{y}{e}{\A\,t_r}/λx\,k.\,t⟩ & \text{handler w/ return clause}\\
	\end{align*}
	\caption{Shorthands (macro-expressions).}
	\label{shorthands}
\end{figure}
\begin{figure}
%	$$L ::= □ │ \Let{x}{e}{L}$$
	\begin{align*}
		(λx.\,e)\,v &→ e\subst{x}{v} & λ.v \\
		\Let{x}{v}{t} &→ t\subst{x}{v} & let.v \\
		⟨v/u⟩ &→ v & d.v \\
		\Let{x}{\S κ.\,e}{t} &→ \S κ.\,e\subst{κ}{κ[\Let{x}{□}{t}]} & let.\S \\
		⟨\S κ.\,e / v⟩ &→ e\subst{κ}{⟨□/v⟩} & d.\S \\
		κ⟦\S κ'.\,e⟧ &→ e\subst{κ'}{κ} & k.\S \\
		⟨L[\S κ.\,e]/v⟩ &→ ⟨L[A\,e\subst{κ}{⟨□/v⟩}]/v⟩ & dL.\S \\
		⟨L[\A\,⟨e/v⟩]/v⟩ &→ ⟨L[e]/v⟩ & \A.d \\
		⟨L[\A\,u]/v⟩ &→ ⟨L[u]/v⟩ & \A.v \\
		\Let{x}{\Let{y}{e}{t_1}}{t_2} &→ \Let{y}{e}{\Let{x}{t_1}{t_2}} & let.let
	\end{align*}
	\caption{Reduction.}
	\label{reduction}
\end{figure}


The grammar of the calculus is in figure \ref{syntax}.
We base it on a fine-grained call-by-value $λ$-calculus,
meaning that the subterms of application $v\;u$ are values
and computations have to be sequenced with $\KwLet$.
The expression $\S κ.\,e$ captures the continuation as $κ$,
a generalization of $\shiftz$.\footnote{
$\S$ is $μ$ in the $λμ$-calculus.
}
The \textit{reset} or \textit{delimiter} $⟨e/v⟩$ delimits continuations captured in $e$ and carries a value $v$.
The \textit{plug} construct $κ⟦e⟧$ installs continuation $κ$ and resumes with execution of $e$.%
%We use the notation $κ[e]$ for plugging the expression $e$ into a continuation $κ$.%
\footnote{
The $λμ$-calculus conventionally uses the notation $⟦κ⟧e$, but we abandon it
as it's inconsistent with the common metanotation of plugging an expression into a context.
}
The value $\Ask (κ)$ refers to the value carried by the delimiter which delimited $κ$.

Contexts are terms with a hole ($□$), with $C[e]$ denoting substituting (or plugging)
the expression $e$ for the hole in $C$.
In general, this is variable-capturing substitution,
but not for $E$ and $K$ since they don't bind variables.
We can also plug a context into a context, as in $C[C']$,
to form another context.
Evaluation contexts $E$ and metacontinuations $K$ use the $□$ to mark the \textit{evaluation position},
whereas tail contexts $L$ mark the \textit{tail position}.
Intuitively, they are the first and the last places where execution occurs.

\section{Structural substitution}

We will be using the notions of
substitution for variables ($e\subst{x}{v}$)
and substitution for continuation variables ($e\subst{κ}{K}$),
the latter of which requires some explaining.
It suffices to consider the two expression formers that use $κ$:
\begin{enumerate}
	\item
		$κ⟦e⟧\subst{κ}{K} ≡ K[e\subst{κ}{K}]$ \\
		On the left, the brackets are on the object level, while on the right they
		denote plugging the expression $e$ into context $K$ on the meta level.
%		This justifies the punning – one can see the $κ⟦e⟧$ as the point where
%		the meta-operation of plugging gets stuck, a normal form.
	\item
		$\Ask(κ)\subst{κ}{⟨K/v⟩} ≡ v$ \\
		$\Ask(κ)\subst{κ}{κ'⟦K⟧} ≡ \Ask(κ')$ \\
		The two cases above cover all substitutions we will be considering.
\end{enumerate}
As further illustration, 
here are the three forms of structural substitution used in reduction (figure \ref{reduction}):
%Reduction (figure \ref{reduction}) uses only three specific forms of structural substitution:
%We also show all the forms of substitution used in reduction (figure \ref{reduction}):
\begin{align*}
	κ⟦e⟧\subst{κ}{κ'} &≡ κ'⟦e\subst{κ}{κ'}⟧ \\
	\Ask(κ)\subst{κ}{κ'} &≡ \Ask(κ') \\
	κ⟦e⟧\subst{κ}{κ⟦\Let{x}{□}{t}⟧} &≡ κ⟦\Let{x}{e\subst{κ}{κ⟦\Let{x}{□}{t}⟧}}{t}⟧ \\
	\Ask(κ)\subst{κ}{κ⟦\Let{x}{□}{t}⟧} &≡ \Ask(κ) \\
	κ⟦e⟧\subst{κ}{⟨□/v⟩} &≡ ⟨e\subst{κ}{⟨□/v⟩}/v⟩ \\
	\Ask(κ)\subst{κ}{⟨□/v⟩} &≡ v
\end{align*}

\section{Evaluation steps}

The evaluation step relation $↦$ is taken to be the closure of rules in figure \ref{step}
under metacontinuations.
The third rule pops the delimiter once computation inside is finished
and the next two rules are responsible for capturing the delimited continuation
when an $\S$ is in evaluation position.
To an extent we allow open evaluation,
namely evaluating under continuation variables and capturing of continuation variables.

We could have opted for fine-grained evaluation steps,
those would be exactly the first six reduction rules (figure \ref{reduction}).
We have taken the coarse-grained steps as the operational semantics because
this greatly simplifies adequacy and it's closer in spirit to machine implementations.

We haven't formalized an abstract machine, but we will be referring to the intuitions that
metacontinuations correspond to machine stacks and $\S$ captures the stack up to the closest reset frame.

\section{Shorthands}
We will be using the shorthands in figure \ref{shorthands}.
The first two of them are commonly used for convenience.
%The first few shorthands (figure \ref{shorthands}) are rather common and just for convenience.
The third one introduces the notation $\A$ for $\abort$.
The fourth one lets us omit the data in delimiters ($/v$) when irrelevant.
%Later, we have macro-expressions that formally justify that the calculus 
Then we have macros that formally justify that the calculus can express the relevant
control operators in their standard form. % TODO style
It is easy to check that evaluation steps simulate the standard evaluation rules for the simulated constructs,
%This is done by checking that evaluation steps simulate the standard evaluation rule for the simulated construct.
%The calculations are easy,
we write out some examples for $\shiftz$ and effect handlers.
\begin{align*}
	⟨E[\shiftz\,k.\,e]⟩
	&≡ ⟨E[\S κ.\,\Let{k}{λx.\,κ⟦x⟧}{e}]⟩ \\
	&↦ \Let{k}{λx.\,⟨E[x]⟩}{e} ↦ e\subst{κ}{λx.\,⟨E[x]⟩}
\end{align*}
\begin{align*}
	\Handle E[\Op v] \With x\,k.\,t
	&≡ ⟨E[\S κ.\,\Ask(κ)\;v\;(λx.\,κ⟦x⟧)]/λx\,k.\,t⟩ \\
	&↦ (λx\,k.\,t)\;v\;(λx.\,⟨E[x]/λx\,k.\,t⟩) \\
	&↦ (λk.\,t\subst{x}{v})\;(λx.\,⟨E[x]/λx\,k.\,t⟩) \\
	&↦ t\subst{x}{v}\subst{k}{λx.\,⟨E[x]/λx\,k.\,t⟩} \\
	&≡ t\subst{x}{v}\subst{k}{λx.\,\Handle E[x] \With x\,k.\,t}
\end{align*}
\begin{align*}
	\Handle v \With x\,k.\,t_h │ x.\,t_r
	&≡ ⟨\Let{x}{v}{\A\,t_r}/λx\,k.\,t_h⟩ \\
	&↦ ⟨\A\,t_r\subst{x}{v}/λx\,k.\,t_h⟩ ↦ t_r\subst{x}{v}
\end{align*}

The last calculation above features a handler with a \textit{return clause}.
The return clause extends the delimited continuation
with a fragment that doesn't run inside the delimiter.
The same extension is possible for reset, there often called \textit{dollar}.
In previous work it has always been an integral part of syntax,
we notice that it can be easily expressed as a $\KwLet$ frame that aborts the delimiter.

\section{Reduction}
We define reduction $→$ to be the closure of rules in figure \ref{reduction}
under general contexts.
We will now demonstrate their applications to optimization,
we may assume common features such as arithmetic for presentation purposes.

%The first few rules are shared with evaluation, now we can use them for simplifications anywhere in the expression.

\subsection{Structural reduction}

The rule $let.\S$ has utility beyond simulating the coarse-grained evaluation steps.
Consider the following as a function body:
$$ \Let{x}{\S κ.\,κ⟦40⟧}{x+2} → \S κ.\,κ⟦\Let{x}{40}{x+2}⟧ → \S κ.\,κ⟦40+2⟧ → \S κ.\,κ⟦42⟧$$
There is no delimiter to find for the $\S$, but we do know a prefix of the continuation
that $\S$ may capture, and we can put this knowledge to good use.

The above example also shows the power of the $κ[e]$ expression former.
We can consider a similar example with the value-capturing $\shiftz$ syntax:
$$\Let{x}{\shiftz\,k.\,k\,40}{f\,x}$$
If $\shiftz$ captures the continuation, $f$ will be applied to $40$, but there is nothing we can do with that knowledge in this syntax.
We could try the following:
$$\shiftz\,k.\, k\,(f\,40)$$
But this is incorrect: $f\,40$ won't run inside the captured continuation, it will be evaluated separately and
then the result will be applied to $k$.
This changes the behavior if $f$ also uses $\shiftz$.

The rule $d.\S$ deals with the easy case where an $\S$ is in direct contact with a delimiter.
The rule $k.\S$, \textit{renaming} in $λμ$-parlance, is more interesting,
since it doesn't have a counterpart within the conventional
continuation-as-lambda syntax.
Consider the following code fragment with consecutive uses of effects:
\begin{align*}
	\Let{\_}{\Op 0}{\Op 1}
	&≡ \Let{\_}{\S κ.\,\Ask(κ)\,0\,(λy.\,κ[y])}{\S κ.\,\Ask(κ)\,1\,(λz.\,κ[z])} & \\
	&→ \S κ.\,\Ask(κ)\,0\,\big(λy.\,κ[\Let{\_}{y}{\S κ.\,\Ask(κ)\,1\,(λz.\,κ[z])}]\big) & let.\S \\
	&→ \S κ.\,\Ask(κ)\,0\,\big(λ\_.\,κ[\S κ.\,\Ask(κ)\,1\,(λz.\,κ[z])]\big) & let.v \\
	&→ \S κ.\,\Ask(κ)\,0\,\big(λ\_.\,\Ask(κ)\,1\,(λz.\,κ[z])\big) & k.\S
\end{align*}
We have completely eliminated the second control operation,
drawing on the fact that both of them will reach the same handler.

There are other natural uses of the plug construct $κ[e]$.
For example, in OCaml 5, delimited continuations are clearly not functions, since
we can not only \textit{continue} them (apply to a value), but also \textit{discontinue} them: raise an exception inside.
This is trivial to express with plug: $\textsf{discontinue}\,κ\,\textit{exn} ≡ κ⟦\textsf{raise}\,\textit{exn}⟧$,
whereas previous formalizations have used workarounds with lambda abstractions.

\subsection{Tail-of-reset reductions}
%By the $let.let$ and $let.\S$ rules, every $\S$
%directly inside a delimiter can be moved into its tail position.
%There, we can apply the $dL.\S$ rule that says this is the delimiter
%that the $\S$ may capture.

The rule $dL.\S$ says that an $\S$ in the tail position of a delimiter
will capture that delimiter (if we get to it).
This is thanks to being in the setting of \textit{static} delimited control.
The rule is obviously infinitely-looping,
but we can prevent that with a side condition that $κ$ actually occurs in $e$.
The rewrite is \textit{at a distance}, as the pattern involves a context $L$.
The change is however local as the outer $⟨L/v⟩$ context remains intact,
so we believe the rule still deserves to be called fine-grained.

We will try to convey the intuition why it's correct by considering the case
where $L$ is just one $\KwLet$ expression: $L ≡ \Let{x}{e}{□}$.
If $e$ evaluates to a value, then the $\S$ will capture the delimiter:
\begin{align*}
	⟨\Let{x}{v}{\S κ.\,t}⟩ ↦ ⟨\S κ.\,t\subst{x}{v}⟩ ↦ t\subst{x}{v}\subst{κ}{⟨□⟩}
\end{align*}
If continuation capture occurs inside $e$,
then the focused $\S$ will still be in the tail position
of the delimiter, now captured. Loosely speaking, we can reason recursively.%
\footnote{
	Perhaps this idea could be formalized as (co)induction on some form of interaction tree,
	with $e$ evaluating to a value as leaves and continuation capture
	resulting in a branch for every expression plugged.
}
\begin{align*}
	⟨\Let{x}{E[\S κ'.\,e']}{\S κ.\,t}⟩ ↦ e'\subst{κ'}{⟨\Let{x}{E}{\S κ.\,t}⟩}
\end{align*}
If the $e$ never evaluates to a value, then the rule rewrites dead code, which is fine.

%The rule is more useful when the delimiter carries useful data,
%as in the case of effect handlers.
%Consider this program that checks satisfiability of Boolean function $f$:
%\begin{align*}
%	\Handle \Let{x}{\Op()}{\Let{y}{\Op()}{f(x,y)}} \With x\,k.\,k\,\true ││ k\,\false \\
%≡ ⟨ \Let{x}{\S κ.\,\Ask(κ)\,()\,(λz.\,κ[z])}{\Let{y}{\Op()}{f(x,y)}} / λx\,k.\,k\,\true ││ k\,\false⟩ \\
%≡ ⟨ \S κ.\,\Ask(κ)\,()\,(λz.\,κ[\Let{x}{z}{\Let{y}{\Op()}{f(x,y)}}]) / λx\,k.\,k\,\true ││ k\,\false⟩
%\end{align*}

With $let.\S$ and $dL.\S$ we can derive the coarse-grained rewrites that started it all:
$$⟨L[E[\shiftz\,k.\,e]]⟩ →^* ⟨L[\abort\,e\subst{k}{λz.\,⟨E[z]⟩}]⟩$$
The one for statically handling effects indeed won't fit on one line:
\begin{alignat*}{2}
	&\Handle L[E[\Op\,v]{}&] \With x\,k.\,t \\
	→^* \quad &\Handle L[\abort\,t\subst{x}{v}\subst{k}{λz.\,\Handle E[z] \With x\,k.\,t}{}&] \With x\,k.\,t
\end{alignat*}


\subsection{Eta-like reductions}

Inspired by the $λμ$-calculus, we could consider an $η$-reduction of the following form:
$$\S κ.\,κ[e] → e$$
In words, capturing a continuation and then immediately installing it
amounts to doing nothing.
However, the rewrite changes program behavior if the program originally got stuck on searching for a delimiter.
We therefore perform the reduction in tail-of-reset contexts to ensure this doesn't happen.
The resulting rule $\A.d$ can be of a slightly simplified form,
since the $dL.\S$ reduction always applies:
$$⟨L[\S κ.\,κ⟦e⟧]⟩ → ⟨L[\A\,⟨e\subst{κ}{⟨□⟩}⟩]⟩$$

The $\A.v$ rule is explained thus: in the tail position of a delimiter,
aborting and returning a value amounts to the same as simply returning a value,
since a delimiter is popped if the inside computation finished with a value.
It is necessary for confluence, specifically to complete this peak:
$$⟨L[\A\,v]⟩ \xleftarrow{d.v} ⟨L[\A\,⟨v⟩]⟩ \xrightarrow{\A.d} ⟨L[v]⟩$$

A common pattern where a captured continuation is immediately resumed is
the \textit{reader} effect,
where an operation (with no payload) asks for a value and the handler provides it:
\begin{align*}
	\Handle e \With \_\,k.\,k\,v
\end{align*}
The reader effect simulates dynamic binding,
with the operation taking the role of reference to a variable and the handler taking the role of the binder.

In general, we don't need to adhere to the protocol that the value carried in the delimiter
is the clause of an effect handler.
We can accomplish the same in a simpler way by putting the dynamically bound value directly
as the data in the delimiter.
To obtain the value in the delimiter we simply do $\textsf{ask} ≡ \S κ.\,κ⟦\Ask(κ)⟧$.
The reduction rules let us replace the $\textsf{ask}$ with the correct value
if it can be statically determined:
\begin{align*}
	⟨\Let{x}{e}{\Let{y}{\textsf{ask}}{t}}/v⟩
	&≡ ⟨\Let{x}{e}{\Let{y}{\S κ.\,κ⟦\Ask(κ)⟧}{t}}/v⟩ \\
	&→ ⟨\Let{x}{e}{\S κ.\,κ⟦\Let{y}{\Ask(κ)}{t}⟧}/v⟩ & let.\S \\
	&→ ⟨\Let{x}{e}{\A\,⟨\Let{y}{v}{t}/v⟩}/v⟩ & dL.\S \\
	&→ ⟨\Let{x}{e}{\Let{y}{v}{t}}/v⟩ & \A.d
\end{align*}
Naturally, the same applies to the effect handler approach.

\subsection{Let reassociation}

The purpose of let reassociation ($let.let$) is to unblock other reductions.
It is one of the ways to obtain a richer theory of open call-by-value \cite{open}.
%Preventing premature $β$-normal forms is one of the ways to obtain a richer theory of open call-by-value \cite{open}:
\begin{align*}
\Let{x}{\Let{y}{e}{v}}{t}
&→ \Let{y}{e}{\Let{x}{v}{t}} & let.let \\
&→ \Let{y}{e}{t\subst{x}{v}} & let.v
\end{align*}
For this work, it's especially useful to unblock the $let.\S$ and tail-of-reset reductions:
\begin{align*}
⟨\Let{x}{\Let{y}{e}{\S κ.\,e}}{t}⟩
&→ ⟨\Let{y}{e}{\Let{x}{\S κ.\,e}{t}}⟩ & let.let \\
&→ ⟨\Let{y}{e}{\S κ.\,e\subst{κ}{κ⟦\Let{x}{□}{t}⟧}}⟩ & let.\S \\
&→ ⟨\Let{y}{e}{\A \,e\subst{κ}{⟨\Let{x}{□}{t}⟩}}⟩ & dL.\S
\end{align*}
More abstractly, it commutes evaluation frames and tail frames:
$$E[L] →^* L[E]$$
%In this way, it improves applicability of the ``monstrous'' coarse-grained reduction.
In this way, it improves applicability of the coarse-grained reductions derived previously:
we can statically handle any effect separated from the handler only by $\KwLet$ expressions.


\chapter{Adequacy}

Our examples in the previous chapter have demonstrated the usefulness
of reduction for optimization.
Now we have to show that the optimization doesn't change the result of a program.


\section{Internal parallel reduction and standard reduction}
In studying interplay of evaluation and reduction,
it will be useful to isolate reductions that are \textit{not} evaluation steps – \textit{internal} reductions.
Furthermore, in later inductive arguments it will be useful that rewrite steps do not multiply.
To this end, we will define \textit{internal parallel reduction}, which can perform internal reductions in many places of the term at once.
This lets us define \textit{standard} reduction as the composition of many evaluation steps and
an internal parallel reduction: $↦^* · \Rrightarrow$.

\section{Postponement}

\begin{theorem}[Strong postponement]
	$${\Rrightarrow · ↦} ⊆ {↦^* · \Rrightarrow}$$
\end{theorem}
\begin{proof}
	By induction on $\Rrightarrow$ and cases.
\end{proof}
\begin{corollary}
	${\Rrightarrow · ↦^*} ⊆ {↦^* · \Rrightarrow}$
\end{corollary}
\begin{proof}
	Induction on the length of $↦^*$.
\end{proof}

We don't need the full factorization (a.k.a.\,semi-standardization) property
$$→^* = (↦ ∪ \Rrightarrow)^* ⊆ {↦^* · \Rrightarrow^*}$$

\section{Commutation}

\begin{theorem}
 \label{quasisubcomm}
	$${\Lleftarrow · ↦ } ⊆ {↦^= · \Lleftarrow · \mapsfrom^*}$$
\end{theorem}
\begin{proof}
	By induction on $\Lleftarrow$ and cases.
\end{proof}
The reflexive closure of $↦$ ($↦^=$, or $≤1$ step of $↦$) appears because of the case
where $\A.d$ and $↦$ do the same thing: $⟨\A\,⟨e⟩⟩$.

\begin{corollary}[Subcommutation] \item
	${(\Lleftarrow · \mapsfrom^*) · ↦} ⊆ {↦^= \mathbin{·} ({{\Lleftarrow} · \mapsfrom^*})}$
\end{corollary}
\begin{proof}
	From the theorem % ${\Lleftarrow · ↦ } ⊆ {↦^= · \Lleftarrow · \mapsfrom^*}$
	and
	${(\Lleftarrow · \mapsfrom^+) · ↦} ⊆ {{{\Lleftarrow} · \mapsfrom^*}}$ (by determinism of $↦$).
\end{proof}
\begin{corollary}
	${(\Lleftarrow · \mapsfrom^*) · ↦^*} ⊆ {↦^* \mathbin{·} ({{\Lleftarrow} · \mapsfrom^*})}$
\end{corollary}
\begin{proof}
	Induction on the length of $↦^*$.
\end{proof}

We probably could have showed ${← · ↦} ⊆ {↦^= · ←^*}$ directly, but we haven't checked.
Having the substitution lemmas already established probably helped.



%\begin{theorem}[Standardization]
%\end{theorem}
%\begin{corollary} \label{stanv}%
%	If $e →^* v$, then $e ↦^* v' →_i^* v$.
%\end{corollary}
%\begin{proof}
%	By standardization and since $→_i$ can't turn a nonvalue into a value.
%\end{proof}
\begin{lemma}
	${→} ⊆ {↦ ∪ \Rrightarrow}$
\end{lemma}

\begin{theorem}[Adequacy]
	If $e → e'$, then $e$ evals to a value iff $e'$ evals to a value.
\end{theorem}
\begin{proof}

%%	We take the proof from Crary.
%	If $e' ↦^* v$, then $e →^* v$. By corollary \ref{stanv} we have $e ↦^* v' →_i^* v$.
%
%	If $e ↦^* v$, then by confluence we have $v →^* v' ←^* e'$, where $v'$ must be a value
%	because $→$ preserves valueness. By corollary \ref{stanv} we have $e' ↦^* v'' →_i^* v'$.
	The case $e ↦ e'$ is easy, we consider $e \Rrightarrow e'$.

	\begin{enumerate}
		\item
			Assume $e ↦ e_1 ↦ ... ↦ e_n = v$.
			We prove $e'$ evals to a value by induction on $n$.

			If $e=e_n=v$, then $e'$ is a value since $\Rrightarrow$ preserves valueness.

			Otherwise, apply subcommutation for $e' \Lleftarrow e ↦ e_1$.
			We obtain $e' ↦^? e'_1 \Lleftarrow · \mapsfrom^* e_1$ for some $e'_1$.
			Since evaluation steps are deterministic and not possible beyond $e_n$,
			we have $e_i \Rrightarrow e'_1$ for some $i$ s.t.\,$1 ≤ i ≤ n$.
			By induction for $e'_1 \Lleftarrow e_i ↦^* v$ we get $e'_1 ↦^* v'$ for some value $v'$.
			Therefore $e' ↦ e'_1 ↦^* v'$.

		\item
			Assume $e' ↦ e'_1 ↦ ... ↦ e'_n = v$.
			We prove $e$ evals to a value by induction on $n$.

			If $e'=e'_n=v$, then $e$ is a value since $\Rrightarrow$ can't make a nonvalue a value.

			Otherwise, apply postponement for $e \Rrightarrow e' ↦ e'_1$.
			We obtain $e ↦^* e'' \Rrightarrow e'_1$ for some $e''$.
			By induction for $e'' \Rrightarrow e'_1 ↦^* v$ we know that $e''$ evals to a value,
			so $e$ also does.

	\end{enumerate}
\end{proof}

%In the above, instead of confluence ${←^* · →^*} ⊆ {→^* · ←^*}$ we could have used
%commutation ${← · ↦^*} ⊆ {↦^* · ←^*}$.
%This could be easier to prove, but we do not explore this approach in this work.

\section{Contextual equivalence}

\begin{corollary}[Contextual equivalence, observing termination to a value]
	\label{ctxeqv1}
	If $e → e'$, then for all general contexts $C$, $C[e]$ evals to a value iff $C[e']$ evals to a value.%
\end{corollary}
\begin{proof}
	By congruence of reduction and adequacy.
\end{proof}

The above notion of contextual equivalence isn't the most common one.
But we can deduce the usual notion from it using the argument of \cite{bisim}.

\begin{prop}[Stuck terms]
\end{prop}

\begin{theorem}[Contextual equivalence, observing termination]
	If $e → e'$, then for all closing contexts $C$, $C[e]$ terminates iff $C[e']$ terminates.
\end{theorem}
\begin{proof}
	By \ref{ctxeqv1} it suffices to show that
	$C[e]$ terminates to a nonvalue iff $C[e']$ terminates to a nonvalue.
\end{proof}



\chapter{Confluence}
Confluence is the property
that diverging reduction sequences can meet.
$$←^* · →^* ⊆ →^* · ←^*$$
Equivalently, it's commutation of $→$ with itself.
In practical terms, it lets us perform reductions in any order
and normal forms are unique.

We haven't managed to prove confluence for
the entire calculus.

We present a proof using the parallel reduction technique for the subcalculus without
data in delimiters and without let reassociation.

\section{Parallel reduction}

\section{The challenges with \textsf{let} reassociation}

The $\KwLet$ reassociation reduction
$$\RLet{x}{\Let{y}{e}{t_1}}{t_2} → \Let{y}{e}{\Let{x}{t_1}{t_2}}$$
flattens the structure of the term by getting rid of $\KwLet$s nested on the left
(here raised).
It has the following critical pair with itself:
$$
\begin{tikzpicture}
	\node (E1) at (2,5) {$\RLet{x}{\RLet{y}{\Let{z}{e}{t_1}}{t_2}}{t_3}$};
	\node (E2) at (6,3) {$\RLet{y}{\Let{z}{e}{t_1}}{\Let{x}{t_2}{t_3}}$};
	\node (E3) at (0,2) {$\RLet{x}{\Let{z}{e}{\Let{y}{t_1}{t_2}}}{t_3}$};
	\node (E3') at (2,1) {$\Let{z}{e}{\RLet{x}{\Let{y}{t_1}{t_2}}}{t_3}$};
	\node (E4) at (4,0) {$\Let{z}{e}{\Let{y}{t_1}{\Let{x}{t_2}{t_3}}}$};
	\path[->]
		(E1) edge (E2)
		(E1) edge (E3);
	\path[->,dashed]
		(E2) edge[bend left] (E4)
		(E3) edge (E3')
		(E3') edge (E4);
	%\draw [brown] (current bounding box.south west) rectangle (current bounding box.north east);
\end{tikzpicture}
$$
Two steps are needed on the left to complete the diagram.
This calls for the following generalization, which can do it in one step and \textit{does} have the diamond property:
$$\Let{x}{L[t_1]}{t_2} → L[\Let{x}{t_1}{t_2}]$$
However, this is not enough for \textit{parallel} reduction.
Here, the redex above overlaps with two different redexes below:
$$
\begin{tikzpicture}
	\node (E1) at (0,0) {$\RLet{x}{\RLet{y}{\RLet{z}{\Let{w}{e}{t_1}}{t_2}}{t_3}}{t_4}$};
	\node (E2) at (5.3,1.1) {$\RLet{x}{\RLet{z}{\Let{w}{e}{t_1}}{\Let{y}{t_2}}{t_3}}{t_4}$};
	\node (E3) at (5.3,-1.1) {$\RLet{y}{\Let{w}{e}{\Let{z}{t_1}}{t_2}}{\Let{x}{t_3}}{t_4}$};
	\path[->] (E1.east) edge[] (E2.south);
	\path[->>] (E1.east) edge[] (E3.north);
	%\draw [brown] (current bounding box.south west) rectangle (current bounding box.north east);
\end{tikzpicture}
$$
To complete this diagram in one parallel reduction step, it would need to have not only length-wise, but also a depth-wise generalization
of $\KwLet$ reassociation.
We don't provide a definition here, since we haven't managed to prove any attempt right.

The other challenge is that a $let.\S$ reduction can tear apart a $let.let$ redex:
$$
\begin{tikzpicture}
	\node (E1) at (0,0) {$\Let{x}{\Let{y}{\S κ.\,e}{t_1}}{t_2}$};
	\node (E2) at (3.5,-2) {$\Let{y}{\S κ.\,e}{\Let{x}{t_1}{t_2}}$};
	\node (E3) at (-3.5,-1.333) {$\Let{x}{\S κ.\,e\subst{κ}{κ[\Let{y}{□}{t_1}]}}{t_2}$};
	\node (E3') at (-3.5,-2.666) {$\S κ.\,e\subst{κ}{κ[\Let{x}{\Let{y}{□}{t_1}}{t_2}]}$};
	\node (E4) at (0,-4) {$\S κ.\,e\subst{κ}{κ[\Let{y}{□}{\Let{x}{t_1}{t_2}}]}$};
	\path[->] (E1) edge (E2);
	\path[->] (E1) edge (E3);
	\path[->,dashed] (E3) edge (E3') (E2) edge (E4);
	\path[->>,dashed] (E3') edge (E4);
\end{tikzpicture}
$$
We need a successive $let.\S$ step to make the $let.let$ redexes whole, now after structural substitution.

\section{Conditional result}


\chapter{Abella mechanization}

We have formalized the adequacy and confluence theorems in the Abella proof assistant\footnote{
	With the yet unreleased bugfix by the author \cite{abellafix}.
}.
Abella follows the tradition of Church's Simple Theory of Types
in representing object-level variable binders as meta-language lambda-abstractions.
This approach to variable binding (known as \textit{HOAS} or \textit{lambda-tree} syntax)
allows reasoning closer to informal pen-and-paper arguments, as it eliminates
much of the noise related to variables and substitution present in other approaches.

\section{Higher-order abstract syntax}

As an example, we represent expressions in the lambda-calculus in Abella by defining constants \lstinline{lam}, \lstinline{app} of the following types:
\begin{lstlisting}
Kind expr   type.
Type lam    (expr -> expr) -> expr.
Type app    expr -> expr -> expr.
\end{lstlisting}
The object-level expression $λx.\,x\,x$ is represented as \lstinline{lam (x\ app x x)},
where \lstinline{x\ ...} is Abella's notation for lambda-abstractions.
We can drop the parentheses and write \lstinline{lam x\ app x x}.
Since the meta-language is typed, strong normalization holds and we can't have troublemakers such
as self-application on the meta level.

With the definition of reduction below, we obtain a \textit{higher-order rewriting system}:
\begin{lstlisting}
Define red : expr -> expr -> prop by
  red (app (lam M) N) (M N);
  red (app M N) (app M' N) := red M M';
  red (app M N) (app M N') := red N N';
  red (lam E) (lam E') := nabla x, red (E x) (E' x).
\end{lstlisting}
The first rule is the $β$-reduction, wherein substitution is realized as application in the meta-language.
Congruence rules defined by Prolog-like clauses follow,
the last of which introduces the \textit{nabla} ($\nabla$) quantifier.

The meta-expression $\nabla x.\,…$ in Abella's logic introduces a fresh nominal constant $x$.
The nabla quantifier has properties intermediate of the classic $∀$ and $∃$ quantifiers.
In practice, nabla lets us operate on object-level terms with free variables:
in the congruence rule for object-level $λ$-abstractions,
we descend under the binder by substituting a fresh nominal constant for the bound variable.
Unlike most formalizations of the $λ$-calculus, we don't have a constant like \lstinline{lam} or \lstinline{app} for variables.
This embodies Perlis' epigram ``There is no such thing as a free variable'':
namely, a variable is always bound
either by the object-level \lstinline{lam x\}
or by the meta-level \lstinline{nabla x}.

Structural substitution poses no problem.
As shown by Abel, continuations can be seen as terms of type \lstinline{expr -> expr} \cite{3rd}.
Plugging an expression \lstinline{e} into a continuation \lstinline{k} (which we have denoted $κ[e]$) is again just
meta-level application: \lstinline{k e}.
Rules such as $let.\S$ transcribe rather directly:
$$\Let{x}{\S κ.\,e}{t} → \S κ.\,e\subst{κ}{κ[\Let{x}{□}{t}]}$$
\begin{center}
\lstinline{red (let (s k\ E k) T) (s k\ E (o\ k (let o T)))}
\end{center}
The main differences being we don't need to name the variable bound in $t$,
but do need to explicitly bind the hole ($□$) as a variable \lstinline{o} in
the substituted extended continuation.
Because of $η$-equivalence in the meta-language,
we can even write \lstinline{s E} instead of \lstinline{s k\ E k} on the left.
The following $β$-equivalence shows that the substitution does the right thing:
\begin{center}
	\lstinline{(o\ k (let o T)) E = k (let E T)}
\end{center}
%This suggests the view of the $κ⟦e⟧$ syntax as the point where the plug meta-operation $K[e]$ gets stuck.

\section{The power of nabla}

HOAS is doesn't need the $\nabla$ quantifier.
Indeed, it's not available in Abella's \textit{specification logic} $λ$-Prolog.
Abella takes a two-level approach, where the weaker specification logic
embeds into the stronger \textit{reasoning logic}.
The lack of $\nabla$ in the specification logic is a source of free theorems (e.g.\, about substitution)
that can be used in the reasoning logic.

Even though the lack of $\nabla$ precludes us from matching on free variables,
there is a general workaround that lets us work with terms with binders:
when we descend under a binder,
we use a \textit{universal} quantifier \lstinline{pi} to represent the bound variable
and add all consequences of it being a variable we will need to the local context of assumptions.
For example, in the definition of parallel reduction for the lambda calculus,
we add the local assumption $x \Rrightarrow x$ in the clause for $λx.\,M \Rrightarrow λx.\,M'$:
\begin{lstlisting}
par (lam M) (lam M') :- pi x\ par x x => par (M x) (M' x).
\end{lstlisting}
In Abella's reasoning logic, we can express the rule $x \Rrightarrow x$ directly:
\begin{lstlisting}
nabla x, par x x;
par (lam M) (lam M') := nabla x, par (M x) (M' x);
\end{lstlisting}

Abel's work on the $λμ$-calculus is in Twelf, and the situation there
is similar to Abella's specification logic: no $\nabla$ quantifier.
Abel noted issues with defining parallel reduction for the $λμ$-calculus in Twelf \cite{3rd}.
There are no issues when we take the $\nabla$ quantifier approach.
Not only is parallel reduction definable,
but, as noted by Accatoli \cite{pearl},
formalizations using only the reasoning logic
can be significantly simpler than using $λ$-Prolog with the above trick,
and so this is the approach we take.
In the end, we believe our work constitutes the first mechanized proof of confluence for a variant of the $λμ$-calculus.

\section{The formalization}
We have introduced the concept of HOAS using $λ$- and $λμ$-calculi as running examples.
Our calculus differs from them in naming and in the separate syntactic category of values.
Moreover, the Abella formalization differs from the definitions in Chapter \ref{Syntax}:
\begin{enumerate}
	\item
		We have an expression $\Ask(K)$ instead of a value $\Ask(κ)$,
		and instead of extending structural substitution we have a reduction rule:
		$$\Ask(⟨K/v⟩) ↦ v$$
		We believe that working with the grammar $\Ask(κ)$ on paper is more convenient,
		but in Abella we wanted to avoid formalizing the normalization to that form.
	\item
		In the Abella formalization, we don't use at-a-distance rules involving contexts, such as:
		$$κ[E[\S κ'.\,e]] ↦ e\subst{κ'}{κ[E]}$$
		A direct transcription of this pattern would create hard unification problems
		involving a unification variable applied to non-variables.
		Furthermore, we would have to introduce a proposition that checks
		that a context \lstinline{expr -> expr} is an evaluation context.
		We instead introduce auxiliary relations that walk the relevant contexts, for example:
		\begin{lstlisting}
nabla k, step (k (E k)) (E' k k) :=
  nabla k, stepc (E k) (E' k);
stepc (shift0 E) E;
stepc (let E T) (k\ E' (o\ k (let o T))) :=
  stepc E E';
		\end{lstlisting}
		That is, \lstinline{stepc} extracts the body of an $\S$ in evaluation position and substitutes the context, semi-formally:
		$$\texttt{stepc}\;E[\S κ.\,e] \;(κ.\,e\subst{κ}{κ[E]})$$
\end{enumerate}

\chapter{Related work}


\section{Distributing delimiters}
There is a strand of work which tackles static delimited control from the opposite direction.
Whereas we use structural reduction so that $\S$ moves closer to the delimiter, that strand's defining feature is
the delimiter distributing over \KwLet{} expressions so that \textit{a} delimiter can meet the control operator:

There, the return clause is an integral part of the syntax. On a sequence of lets this becomes

We consider turning every let into a delimiter unsatisfactory, since delimiters usually have a much higher runtime cost than let bindings.

Because they lack structural reduction, those systems cannot perform the optimizations in examples TODO.
We show that translations from those calculi preserve equations (understood as the equivalence closure of reduction)
to argue that our calculus is more powerful.

\section{shift0 calculus}
By reflection\cite{ppdp21} it follows that the original shift0 equational theory by Materzok is also less powerful,
but for completeness we show it directly.

\section{Delimited \texorpdfstring{$λμ$}{lambda-mu}}

\section{Expressing handlers using shift0}

\section{Fine-grained syntax for algebraic operations}
The original syntax for algebraic operations involved continuations.
What we may now express as the generic effect $arb : Bool$
can be equivalently expresssed as an algebraic operation $or : (unit → τ)^2 → τ$
with the equivalences $arb = or(λ\_.\,true, λ\_.\,false)$ and
$or(k_1,k_2) = if arb then k_1 () else k_2 ()$ \cite{alggen}.

%Handling of algebraic operations can be characterized by equalities \cite{handlers}, viz.
%$$\Handle \Op(x_1.\,t_1,...,x_n.\,t_n) \With k_1,...,k_n.\,t_h =$$

The connection of algebraic operations and control wasn't explicitly noted at first,
but can be seen in the naturality condition for algebraic operations,
which states that evaluation contexts commute with operations \cite{logic, handling}:
$$E[\Op(x_1.\,t_1, ..., x_n.\,t_n)] = \Op(x_1.\,E[t_1], ..., x_n.\,E[t_n])$$

The calculus in \cite{hia} has an intermediate approach,
where an operation has only one continuation parameter.
The intended surface syntax is of the generic effect form $\Op v$,
which is elaborated to start with an identity continuation $\Op v\,(λx.\,x)$.
The step relation captures the evaluation context in a fine-grained manner:
$$\Let{y}{\Op v\,(λx.\,t_1)}{t_2} ↦ \Op v\,(λx.\,\Let{y}{t_1}{t_2})$$
This bears similarity to our approach based on structural reduction.
However, we don't see a way to express our optimizations without introducing $\abort$ in some form.

\section{Confluence}

Proofs of confluence have been notoriously difficult since the beginning --
the many early attempts for the $λ$-calculus were flawed.

The proof alongside the introduction of the $λμ$-calculus \cite{parigot92} also turned out to be incorrect \cite{baba}.
The fix involves performing multiple structural reduction steps \cite{baba,koji}, which we have employed in our formalization
and verified that it works also with our tail-of-delimiter reductions.

Herbelin and Zimmerman \cite{Herbelin} claim that a proof by parallel reduction is possible for a $λμ$-calculus with
$\KwLet$ reassociation, but provide no details. We don't see how they could deal with the $let.\S$-$let.let$ critical pair
unless they add a $κ[\Let{x}{e}{t}] → \Let{x}{e}{κ[t]}$ reduction, which is similar to theirs
$x(\Let{x}{e}{t}) → \Let{x}{e}{xt}$ and seems to be valid in the undelimited setting.

The proof in \cite{ppdp21} is claimed by local confluence of $\Rightarrow$ such that $→^*=\Rightarrow^*$,
which is not a valid argument (take $\Rightarrow=→$ for a counterexample, it is known that local confluence does not imply confluence). Diamond property is needed instead of local confluence.



\section{Tail reductions and letcc}
The central idea in our paper is embodied by the reduction
$$⟨L[\S κ.\,e]⟩ → ⟨L[\A\,e\subst{κ}{⟨□⟩}]⟩$$
It is tempting to decompose $\S$ into its constituent parts:
a $\keyword{letcc}$ which binds the continuation without aborting, and an abort $\A$:
$$\S κ.\,e ≡ \keyword{letcc}\;κ.\,\A\,e$$
Then we could simulate $dL.\S$ using a reduction that pushes $\keyword{letcc}$ out of the tail:
$$\Let{x}{e}{\keyword{letcc}\;κ.\,t} → \keyword{letcc}\;κ.\,\Let{x}{e}{t}$$
Analogous reductions have been considered for undelimited continuations \cite{sabry}.

There are issues with this approach, though:
\begin{enumerate}
	\item
		The above reduction isn't semantics-preserving. If $e$ above is the nonterminating $Ω$,
		then the left side diverges, while the right side gets stuck on searching for a delimiter.
		We could ensure a delimiter is found by performing the reduction only in $⟨L⟩$,
		but then it is unclear if there are serious benefits to this approach.

	\item
		Because $\keyword{letcc}$ binds the continuation and keeps running in it
		$$⟨E[\keyword{letcc}\;κ.\,e]⟩ ↦ ⟨E[e\subst{κ}{⟨E⟩}]⟩$$
		it seems inherently incompatible with one-shot (linear) continuations.
		This is a design choice or a limitation of some implementations of control, e.g.\,OCaml 5.
\end{enumerate}


\section{Expressing return clauses}
Recall our encoding of return clauses:
$$⟨e│x.\,e_r⟩ ≡ ⟨\Let{x}{e}{\A\;e_r}⟩$$
$$\Handle e \With x,k.\,e_h │ x.\,e_r ≡ ⟨\Let{x}{e}{\A\;e_r}│λx,k.\,e_h⟩$$
An equivalence reminiscent of it has been proven for typed algebraic effects \cite{hwc}:
$$\Handle e \With x,k.\,e_h │ x.\,e_r ≈\Handle \Let{x}{e}{\Lift{e_r}} \With x,k.\,e_h$$
Our encoding is clearly the natural one, since its correctness
is proven just by calculation of a few evaluation steps.

The idea in both is that we want $e_r$ to be unable to reach the handler.
After $e$ has finished evaluating and we start evaluating the rhs of $\KwLet$,
we know that the frame at the top of the stack will be the handler.
Our solution is popping the handler with $\abort$,
theirs is pushing a $\KwLift$ stack frame that makes effects skip the $\KwLift$-handler pair \cite[Appendix A]{hwc}.

\chapter{Discussion and future work}

While this work brings forth many new ideas, it is far from being the IR of a compiler.

We believe all results transfer to multi-prompt delimited control, where we have independent sets of labeled operators $S^\ell κ. e$, $⟨e⟩^\ell$.
This is necessary for working with multiple effects, though the lift construct can to some extent simulate that.
Naturally, the purity assertions also could be refined to block or allow specific listed effects.
The multi-prompt calculus need a mild form of typing: in the renaming reduction we need to know that the continuation variable $κ$ comes
from an $\S$ with the matching label
$$κ^\ell[\S^\ell κ'.\,e] → e\subst{κ'}{κ}$$

One could wonder why we don't have tail-of-plug reductions
$$κ[L[\S κ'.\,e]] → κ[L[A\,e\subst{κ'}{κ}]]$$
The reason is they don't work in the multi-prompt setting.
If $κ$ is allowed to have delimiters of different labels in the middle, then
the continuation may not be intact by the time we get to the $\S$.

It should be clear that the central idea of this work, the tail-of-delimiter reductions,
doesn't apply to \textit{dynamic} delimited control.
They are those that make the delimiter disappear after capture, for example \textsf{control\textsubscript{0}}:
$$⟨E[\mathsf{control_0} κ.\,e]⟩ ↦ e\subst{κ}{E}$$
The only hope for optimizations seems to be purity-aware reductions,
which could move the delimiter directly to the operator. The same applies to the related \textit{shallow} effect handlers.
There is however an option between shallow and deep handlers: \textit{sheep} handlers \cite{sheep},
where the delimiter is static,
but the dynamically bound interpretation of the effect can be used at most once
and has to be manually reinstated every time a continuation is installed.




Realistic languages also have more tail contexts such as $if b then □ else □$ as well as evaluation contexts.

join points?

%\hfuzz=1pt
\printbibliography[heading=bibintoc]

\end{document}
