\documentclass{beamer}
\usepackage[utf8]{inputenc}
\usepackage[T1]{fontenc}
\usepackage{silence}
\usepackage{amsmath,amsfonts,amssymb,amsthm}
\usepackage{mathtools}
\usepackage{enumitem}
\usepackage{bm}
\usepackage{newunicodechar}
\usepackage{etoolbox}
\usepackage{stmaryrd}
\usepackage{stmaryrd}
\SetSymbolFont{stmry}{bold}{U}{stmry}{m}{n}
% https://tex.stackexchange.com/a/106719
\DeclareSymbolFont{sfletters}{OML}{cmbrm}{m}{it}
\DeclareMathSymbol{\slambda}{\mathord}{sfletters}{"15}
%https://tex.stackexchange.com/a/335857
\usepackage{mathpartir}
\usepackage{tikz}
\usetikzlibrary{arrows} % For advanced arrow tips


\usepackage[nodisplayskipstretch]{setspace}
\setlength{\parskip}{0pt}


\title{\textbf{A Fine Calculus for Static Delimited Control}}
\author{Wiktor Kuchta}

\date{11 kwietnia 2025}

\newcommand{\Par}[1]{\stackrel{#1}{\Rrightarrow}}
\newcommand{\Rap}[1]{\stackrel{#1}{\Lleftarrow}}
\newcommand{\Int}{\Rightarrow}
\newcommand{\Tni}{\Leftarrow}
\newcommand{\Stan}{↦^* · \Int}
\newcommand{\foreign}[1]{#1}
\newcommand{\tagit}[1]{\tag{\textit{#1}}}
\newcommand{\tagmath}[1]{\tag{\(#1\)}}
\newcommand{\Log}{\textsf{log}}
\newcommand{\true}{\textsf{true}}
\newcommand{\false}{\textsf{false}}
\newcommand{\shiftz}{\textsf{shift0}}
\newcommand{\abort}{\textsf{abort}}
\newcommand{\keyword}[1]{\textsf{\textup{#1}}}
\newcommand{\KwOp}{\keyword{op}}
\newcommand{\Op}{\KwOp\,}
\newcommand{\KwHandle}{\keyword{handle}}
\newcommand{\Handle}{\KwHandle\;}
\newcommand{\KwWith}{\keyword{with}}
\newcommand{\With}{\;\KwWith\;}
\newcommand{\KwRaise}{\keyword{raise}}
\newcommand{\Raise}{\KwRaise\,}
\newcommand{\Ask}{\textsf{ask}}
\newcommand{\KwTry}{\keyword{try}}
\newcommand{\Try}{\KwTry\;}
\newcommand{\KwLet}{\keyword{let}}
\newcommand{\Let}[3]{\keyword{let}\:#1\:\keyword{=}\:#2\:\keyword{in}\:#3}
\newcommand{\Dlet}[3]{\keyword{dlet}\:#1\:\keyword{=}\:#2\:\keyword{in}\:#3}
\newcommand{\Letp}[3]{\keyword{let}^p\:#1\:\keyword{=}\:#2\:\keyword{in}\:#3}
\newcommand{\RLet}[3]{\Let{#1}{\raisebox{0.5 ex}{$#2$}}{#3}}
\newcommand{\Letrec}[3]{\keyword{letrec}\:#1\:\keyword{=}\:#2\:\keyword{in}\:#3}
\newcommand{\KwLift}{\keyword{lift}}
\newcommand{\Lift}[1]{\KwLift\;#1}
\newcommand{\subst}[2]{\{#1{:=}#2\}}
\newcommand{\E}{\mathcal{E}}
\newcommand{\K}{\mathcal{K}}
\renewcommand{\S}{\mathcal{S}}
\newcommand{\A}{\mathcal{A}}
\newcommand{\kT}{\mathsf{T}}
\newcommand{\kE}{\mathsf{E}}
\newcommand{\kR}{\mathsf{R}}
\newcommand{\Free}{\textrm{-}\mathrm{free}}
\newcommand{\Obs}{\mathrm{Obs}}
\newcommand{\N}{\mathbb{N}}
\DeclareMathOperator{\dom}{dom}
\newcommand{\+}{\enspace}
\newcommand{\lStr}{\textsf{Str}}
\newcommand{\lPar}{\textsf{Par}}
\newcommand{\gray}[1]{\textcolor{gray}{#1}}
\newcommand{\blue}[1]{\textcolor{blue}{#1}}

\newunicodechar{≅}{\cong}
\newunicodechar{│}{\mid} % Digr vv
\newunicodechar{╱}{\bm{/}} % Digr FD
\newunicodechar{∷}{::} % Digr ::
\newunicodechar{□}{\square} % Digr OS
\newunicodechar{∅}{\emptyset} % Digr /0
\newunicodechar{α}{\alpha}
\newunicodechar{β}{\beta}
\newunicodechar{δ}{\delta} % Digr d*
\newunicodechar{ε}{\varepsilon}
\newunicodechar{γ}{\gamma} % Digr g*
\newunicodechar{ι}{\iota} % Digr i*
\newunicodechar{κ}{\kappa}
\newunicodechar{λ}{\lambda}
\newunicodechar{Λ}{\Lambda}
\newunicodechar{μ}{\mu}
\newunicodechar{ν}{\nu}
\newunicodechar{ρ}{\rho}
\newunicodechar{σ}{\sigma}
\newunicodechar{τ}{\tau}
\newunicodechar{η}{\eta} % Digr y*
\newunicodechar{Δ}{\Delta}
\newunicodechar{Γ}{\Gamma}
\newunicodechar{Ω}{\Omega} % digr W*
\newunicodechar{ℕ}{\N} % Digr NN 8469 nonstandard
\newunicodechar{⊆}{\subseteq} % Digr (_
\newunicodechar{∪}{\cup} % Digr )U
\newunicodechar{∈}{\in} % Digr (-
\newunicodechar{∃}{\exists} % Digr TE
\newunicodechar{∀}{\forall} % Digr FA
\newunicodechar{∧}{\wedge} % Digr AN
\newunicodechar{∨}{\vee} % Digr OR
\newunicodechar{⊥}{\bot} % Digr -T
\newunicodechar{⊢}{\vdash} % Digr \- 8866 nonstandard
\newunicodechar{⊨}{\models} % Digr \= 8872 nonstandard
\newunicodechar{⊤}{\top} % Digr TO 8868 nonstandard
\newunicodechar{⇒}{\implies} % Digr =>
\newunicodechar{⇔}{\iff} % Digr ==
\newunicodechar{↦}{\mapsto} % Digr T> 8614 nonstandard
\newunicodechar{≠}{\neq}
\newunicodechar{⟦}{\bm{[}}
\newunicodechar{⟧}{\bm{]}}
\newunicodechar{≥}{\ge}
\newunicodechar{≤}{\le}
\newunicodechar{≡}{\equiv}
\newunicodechar{≈}{\approx}
\newunicodechar{●}{\bullet}


% cursed
\WarningFilter{newunicodechar}{Redefining Unicode}
\newunicodechar{·}{\ifmmode\cdot\else\textperiodcentered\fi} % Digr .M
\newunicodechar{×}{\ifmmode\times\else\texttimes\fi} % Digr *X
\newunicodechar{→}{\ifmmode\rightarrow\else\textrightarrow\fi} % Digr ->
\newunicodechar{←}{\ifmmode\leftarrow\else\textleftarrow\fi} % Digr <-
\newunicodechar{…}{\ifmmode\dots\else\textellipsis\fi} % Digr .,
\newunicodechar{⟨}{\bm{\langle}}
\newunicodechar{⟩}{\bm{\rangle}}
\newunicodechar{¦}{\bm{│}}


\begin{document}
\maketitle

\begin{frame}
	\frametitle{Flavors of delimited control}
	\begin{align*}
		&\shiftz\, k.\,e && ⟨□⟩ \\
		\\
		&\Op v && \Handle □ \With x\;k.\,e \\
		\\
		\uncover<2->{&\S κ.\,e && ⟨□╱v⟩}
	\end{align*}
\end{frame}

\begin{frame}
	\frametitle{$\shiftz$ generalized to cont vars}
	\begin{align*}
		⟨\Let{x}{\S κ. \overbrace{\gray{...}\: κ⟦e⟧ \:\gray{...}}^{\text{body}}}{t}⟩
		↦
		\overbrace{\gray{...} \:⟨\Let{x}{e}{t}⟩ \:\gray{...}}^{\text{body after subst}}
	\end{align*}

	\pause
	\textit{Structural substitution}, known from the $λμ$-calculus.

	Formal evaluation rule:
	$$D[\S κ.\,e] ↦ e\subst{κ}{D}$$
\end{frame}

\begin{frame}
	\frametitle{Data in delimiters: dynamic binding}
	\begin{align*}
		⟨{\S κ. \overbrace{\gray{...}\: \Ask(κ) \:\gray{...}}^{\text{body}}}╱v⟩
		↦
		\overbrace{\gray{...} \: v \:\gray{...}}^{\text{body after subst}}
	\end{align*}

	\pause
	Can express effect handlers:
	\begin{align*}
		\Op v &≡ \S κ.\,\Ask(κ)\:v\:(λx.\,κ⟦x⟧)\\
		\Handle e \With x\,k.\,t &≡ ⟨e╱λx\,k.\,t⟩\\
	\end{align*}
	(Previous encodings of handlers using shift0/reset unwittingly encoded dynamic binding.)

\end{frame}

\begin{frame}
	\frametitle{Grammar}
\begin{align*}
	 &&v,u &::= x │ λx.\,e │ \Ask(κ) \\
	 &&e,t &::= v │ v\:v │ \Let{x}{e}{t} │ \S κ.\,e │ ⟨e ╱ v⟩ │ κ⟦e⟧  \\
\end{align*}
\end{frame}

\begin{frame}
	\frametitle{Fine-grained reduction}
	\begin{align*}
		(λx.\,e)\,v &→ e\subst{x}{v} \tagmath{λ.v} \\
		\Let{x}{v}{t} &→ t\subst{x}{v} \tagmath{\KwLet{}.v} \\
		⟨v╱u⟩ &→ v \tagmath{d.v} \\
		\Let{x}{\S κ.\,e}{t} &→ \S κ.\,e\subst{κ}{κ⟦\Let{x}{□}{t}⟧} \tagmath{\KwLet{}.\S} \\
		⟨\S κ.\,e ╱ v⟩ &→ e\subst{κ}{⟨□╱v⟩} \tagmath{d.\S} \\
		κ⟦\S κ'.\,e⟧ &→ e\subst{κ'}{κ} \tagmath{k.\S} \\
		⟨L[\S κ.\,e]╱v⟩ &→ ⟨L[\A\,e\subst{κ}{⟨□╱v⟩}]╱v⟩ \tagmath{dL.\S} \\
		⟨L[\A⟨e╱v⟩]╱v⟩ &→ ⟨L[e]╱v⟩ \tagmath{\A.d} \\
		⟨L[\A\,u]╱v⟩ &→ ⟨L[u]╱v⟩ \tagmath{\A.v} \\
		\Let{x}{\Let{y}{e}{t_1}}{t_2} &→ \Let{y}{e}{\Let{x}{t_1}{t_2}} \tagmath{\KwLet{}.\KwLet{}}
	\end{align*}

	$$L ::= □ │ \Let{x}{e}{L}$$
\end{frame}

\begin{frame}
	\frametitle{Correctness of reduction}
	\tikzset{
  every path/.style={ % Apply to all paths drawn
    ->>,             % Use the double arrow tip
    very thick       % Set line thickness (also makes '->>' tip bigger)
  }
}

\begin{center}
\begin{tikzpicture}
  \coordinate (T) at (0, 1.5);
  \coordinate (R) at (1.5, 0);
  \coordinate (B) at (0, -1.5);
  \coordinate (L) at (-1.5, 0);

  \draw[->>, blue] (T) -- (R);
  \draw[->>, orange] (L) -- (T);
  \draw[->>, dashed, orange] (B) -- (R);
  \draw[->>, dashed, blue] (L) -- (B);

\end{tikzpicture}
\hspace{2cm}
\begin{tikzpicture}
  \coordinate (T) at (0, 1.5);
  \coordinate (TL) at (-0.75, 0.75);
  \coordinate (R) at (1.5, 0);
  \coordinate (B) at (0, -1.5);
  \coordinate (BR) at (0.75, -0.75);
  \coordinate (L) at (-1.5, 0);

  \draw[->>, blue] (T) -- (R);
  \draw[->>, blue] (T) -- (TL);
  \draw[->>, orange] (TL) -- (L);
  \draw[->>, dashed, blue] (R) -- (BR);
  \draw[->>, dashed, orange] (BR) -- (B);
  \draw[->>, dashed, blue] (L) -- (B);

\end{tikzpicture}
\end{center}

Postponement and commutation of \textcolor{blue}{evaluation} and \textcolor{orange}{nonevaluation}.

\end{frame}

\begin{frame}
	\frametitle{Confluence}
	\tikzset{
  every path/.style={ % Apply to all paths drawn
    ->>,             % Use the double arrow tip
    very thick       % Set line thickness (also makes '->>' tip bigger)
  }
}
\begin{center}
\begin{tikzpicture}
  \coordinate (T) at (0, 1.5);
  \coordinate (R) at (1.5, 0);
  \coordinate (B) at (0, -1.5);
  \coordinate (L) at (-1.5, 0);

  \draw[->> ] (T) -- (L);
  \draw[->> ] (T) -- (R);
  \draw[->>, dashed ] (R) -- (B);
  \draw[->>, dashed ] (L) -- (B);

\end{tikzpicture}
\end{center}

Now considering any pairs of reductions: much harder.

\end{frame}

\begin{frame}

	\frametitle{Abella formalization}
	Higher-order encoding of continuation variables:
	$$\texttt{Type shift0     ((expr -> expr) -> expr) -> expr.}$$

	First mechanization of parallel reductions and rewriting properties for $λμ$-calculus.

\end{frame}

\begin{frame}
	\frametitle{Explain and improve upon distributing delimiters}
\begin{align*}
⟨\Let{x}{e}{t} ¦ y.\,t_r⟩
&≡ ⟨\Let{y}{\Let{x}{e}{t}}{\A\,t_r}⟩ \\
&→ ⟨\Let{x}{e}{\Let{y}{t}{\A\,t_r}}⟩ \\
&← ⟨\Let{x}{e}{\A⟨\Let{y}{t}{\A\,t_r}⟩}⟩ \\
&≡ ⟨e ¦ x.\, ⟨t ¦ y.\,t_r⟩⟩
\end{align*}

\end{frame}

\end{document}
